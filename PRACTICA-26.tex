\documentclass[12pt,letterpaper]{article}\usepackage[]{graphicx}\usepackage[]{color}
%% maxwidth is the original width if it is less than linewidth
%% otherwise use linewidth (to make sure the graphics do not exceed the margin)
\makeatletter
\def\maxwidth{ %
  \ifdim\Gin@nat@width>\linewidth
    \linewidth
  \else
    \Gin@nat@width
  \fi
}
\makeatother

\definecolor{fgcolor}{rgb}{0.345, 0.345, 0.345}
\newcommand{\hlnum}[1]{\textcolor[rgb]{0.686,0.059,0.569}{#1}}%
\newcommand{\hlstr}[1]{\textcolor[rgb]{0.192,0.494,0.8}{#1}}%
\newcommand{\hlcom}[1]{\textcolor[rgb]{0.678,0.584,0.686}{\textit{#1}}}%
\newcommand{\hlopt}[1]{\textcolor[rgb]{0,0,0}{#1}}%
\newcommand{\hlstd}[1]{\textcolor[rgb]{0.345,0.345,0.345}{#1}}%
\newcommand{\hlkwa}[1]{\textcolor[rgb]{0.161,0.373,0.58}{\textbf{#1}}}%
\newcommand{\hlkwb}[1]{\textcolor[rgb]{0.69,0.353,0.396}{#1}}%
\newcommand{\hlkwc}[1]{\textcolor[rgb]{0.333,0.667,0.333}{#1}}%
\newcommand{\hlkwd}[1]{\textcolor[rgb]{0.737,0.353,0.396}{\textbf{#1}}}%

\usepackage{framed}
\makeatletter
\newenvironment{kframe}{%
 \def\at@end@of@kframe{}%
 \ifinner\ifhmode%
  \def\at@end@of@kframe{\end{minipage}}%
  \begin{minipage}{\columnwidth}%
 \fi\fi%
 \def\FrameCommand##1{\hskip\@totalleftmargin \hskip-\fboxsep
 \colorbox{shadecolor}{##1}\hskip-\fboxsep
     % There is no \\@totalrightmargin, so:
     \hskip-\linewidth \hskip-\@totalleftmargin \hskip\columnwidth}%
 \MakeFramed {\advance\hsize-\width
   \@totalleftmargin\z@ \linewidth\hsize
   \@setminipage}}%
 {\par\unskip\endMakeFramed%
 \at@end@of@kframe}
\makeatother

\definecolor{shadecolor}{rgb}{.97, .97, .97}
\definecolor{messagecolor}{rgb}{0, 0, 0}
\definecolor{warningcolor}{rgb}{1, 0, 1}
\definecolor{errorcolor}{rgb}{1, 0, 0}
\newenvironment{knitrout}{}{} % an empty environment to be redefined in TeX

\usepackage{alltt}
 \usepackage[left=2cm,right=2cm,top=2cm,bottom=2cm]{geometry}
\usepackage[ansinew]{inputenc}
\usepackage[spanish]{babel}
\usepackage{amsmath}
\usepackage{amsfonts}
\usepackage{amssymb}
\usepackage{dsfont}
\usepackage{multicol} 
\usepackage{subfigure}
\usepackage{graphicx}
\usepackage{float} 
\usepackage{verbatim} 
\usepackage[left=2cm,right=2cm,top=2cm,bottom=2cm]{geometry}
\usepackage{fancyhdr}
\pagestyle{fancy} 
\fancyhead[LO]{\leftmark}
\usepackage{caption}
\newtheorem{definicion}{Definci\'on}
\IfFileExists{upquote.sty}{\usepackage{upquote}}{}
\begin{document}

\begin{titlepage}
\setlength{\unitlength}{1 cm} %Especificar unidad de trabajo


\begin{center}
\textbf{{\large UNIVERSIDAD DE EL SALVADOR}\\
{\large FACULTAD MULTIDISCIPLINARIA DE OCCIDENTE}\\
{\large DEPARTAMENTO DE MATEM\'ATICA}}\\[0.50 cm]

\begin{picture}(18,4)
 \put(7,0){\includegraphics[width=4cm]{minerva.jpg}}
\end{picture}
\\[0.25 cm]

\textbf{{\large Licenciatura en Estad\'istica}\\[1.25cm]
{\large Control Estadistico del Paquete R }\\[2 cm]
%\setlength{\unitlength}{1 cm}
{\large  \textbf{''UNIDAD SEIS"}}\\[3 cm]
{\large Alumna:}\\
{\large Erika Beatr\'iz Guill\'en Pineda}\\[2cm]
{\large Fecha de elaboraci\'on}\\
Santa Ana - \today }
\end{center}
\end{titlepage}

\newtheorem{teorema}{Teorema}
\newtheorem{prop}{Proposici\'on}[section]


\lhead{PR\'ACTICA 26}
\lfoot{LICENCIATURA EN ESTAD\'ISTICA}
\cfoot{UESOCC}
\rfoot{\thepage}
%\pagestyle{fancy} 

\setcounter{page}{1}
\newpage
\section{INTRODUCCI\'ON}
Los dise\~nos bifactoriales se diferencian a los dise\~nos por bloques en que ahora si nos interesa conocer el efecto entre este segundo factor(variable bloque) y nuestra variable dependiente; y adem\'as los dos factores ya no son independientes, por lo que es necesario conocer tambi\'en el efecto de la interacci\'on de ambos factores en nuestra variable dependiente.\\

Supondremos que tenemos un primer factor, A, con k niveles; mientras que tenemos un segundo factor, B, con n niveles. Supondremos que tomamos para cada combinaci\'on posible de los niveles de ambos factores un total de m observaciones. De lo contrario tendremos m\'as par\'ametros a estimar en el modelo que el n\'umero de observaciones disponibles; y por consiguiente ser\'a imposible realizar el contraste.\\

El modelo que genera los datos es el siguiente:\\

$y_ \ ijl$ = $\µ$ + $t_i$ + $B_j$ + $($tB$)_ \ij$ + $u_\ ijl$\\

Donde:
\begin{itemize}
  \item $y_ \ ijl$: Representa la l-\'esima observaci\'on para la combinaci\'on ij-\'esima de niveles de los factores A y B (celda ij-\'esima de la matriz de datos). 
  \item $\µ$: Representa un promedio o efecto global. 
  \item $t_i$: Representa el efecto del factor A cuando \'este se encuentra en el i-\'esimo nivel. Debe cumplirse que la sumatoria de $t_i$ = 0.
  \item $B_j$: Representa el efecto del factor B cuando \'este se encuentra en el j-\'esimo nivel. Deben cumplir que la sumatoria de $B_j$ = 0.
  \item  $($tB$)_ \ij$: Representa el efecto de la interacci\'on de los factores A y B cuando el primero se encuentra en el nivel i y el segundo en el nivel j.
  \item $u_\ ijl$: Representa un componente de error aleatorio, llamado perturbaciones, que incorpora todas las dem\'as fuentes de variabilidad del experimento.
\end{itemize}

Las cuatro hip\'otesis b\'asicas del modelo se resumen como siempre en\\
$u_ \ ijl$ = $NIID$ N(0; $sigma^2$) =; para todo i,j,l.\\

El primer contraste de hip\'otesis a realizar es (efecto del factor A):\\

$H_o$: $t_1$ = $t_2$ ... $t_k$ = 0\\

$H_1$: $t_i$ distinto 0; para al menos un $i$\\

El segundo contraste de hip\'otesis a realizar es (efecto del factor B):\\

$H_o$: $B_1$ = $B_2$ ... $B_k$ = 0\\

$H_1$: $B_j$ distinto 0; para al menos un $j$\\

Mientras que el tercer contraste de hip\'otesis a realizar es (efecto de la interacci\'on de los factores A y B):\\

$H_o$: $($tB$)_ \ij$ = 0; para todo i,j\\

$H_1$: $($tB$)_ \ij$ disntinto 0 para al menos una combinaci\'on ij\\

No resulta dif\'icil verificar utilizando el m\'etodo de m\'axima verosimilitud que el modelo estimado para una muestra aleatoria de tama\~no Nkn = es:\\

$\^y_ \ ij$ = $\^\µ$ + $\^t_i$ + $\^B_j$ + $\^ ($alfaB$)_\ ij\\

Y por consiguiente:\\

$y_ \ ijl$ = $\^\µ$ + $\^t_i$ + $\^B_j$ + $\^ ($alfaB$)_\ij$ + $\^ u_\ ijl$\\

Donde: 
\begin{itemize}
  \item $\^\µ$ = $\~y_..$
  \item $\^t_i$ =  $\~y_i..$ -  $\~y_..$
  \item $\^B_j$ = $\~y_.j.$ -  $\~y_..$
  \item $\^ ($alfaB$)_\ij$ = $\~y_ij.$ -  $\~y_i..$ - $\~y_.j.$ -  $\~y_..$
  \item $\^ u_\ ijl$ = $y_\ ijl$ - $\~y_ij.$
\end{itemize}

Y se tendr\'an las siguientes medidas de inter\'es:

\begin{itemize}
  \item $\~y_i..$ es el promedio para el i-\'esimo nivel del factor A.
  \item $\~y_.j.$ es el promedio para el j-\'esimo nivel del factor B.
  \item $\~y_ij.$ es el promedio para la combinaci\'on ij-\'esima de los factores A y B.
  \item $\~y_..$ es la media general de la caracter\'istica de inter\'es.
\end{itemize}

El An\'alisis de Varianza establece que se debe cumplir la siguiente relaci\'on (al ser cada uno de las fuentes ortogonales entre s\'i):\\

VT = VE(t) + VE(B) + VE(tB) + VNE\\

Donde: 
\begin{itemize}
  \item VT es la variabilidad total del experimento.
  \item VE(t) es la variabilidad explicada por el factor A.
  \item VE(B) es la variabilidad explicada por el factor B. 
  \item VE(tB) es la variabilidad explicada por la combinaci\'on de los factores A y B. 
  \item VNE es la variabilidad no explicada o residual.
\end{itemize}

Para poder contrastar cada una de los diferentes contrastes anteriores, se utiliza la tabla ANOVA siguiente:\\

\textbf{Factor A:}
\begin{itemize}
  \item Sumas de Cuadrados: VE($t$).
  \item Grados de Libertad: $K-1$
  \item Medias de Cuadrados: $MCE($t$)$ = $VE($t$)$ / $K-1$ 
  \item $F_o$: $F_\ t$ = $MCE($t$)$/$MCNE$
\end{itemize}

\textbf{Factor B:}
\begin{itemize}
  \item Sumas de Cuadrados: $VE(B)$.
  \item Grados de Libertad: $n - 1$
  \item Medias de Cuadrados: $MCNE(B)$ = $VE($t$)$ / $k-1$
  \item $F_o$: $F_\ B$ = $MCE(B)$/$MCNE$
\end{itemize}

\textbf{Interacci\'on:}
\begin{itemize}
  \item Sumas de Cuadrados: $VE($tB$)$.
  \item Grados de Libertad: $(K - 1)(n - 1)$
  \item Medias de Cuadrados: $MCNE$ = $VNE($tB$)$ / $(K - 1)(n - 1)$
  \item $F_o$: $F_\ tB$ = $MCE($tB$)$/$MCNE$
\end{itemize}

\textbf{Error:}
\begin{itemize}
  \item Sumas de Cuadrados: $VNE$.
  \item Grados de Libertad: $kn(m - 1)$
  \item Medias de Cuadrados: $MCNE$ = $VNE$ / $kn(m - 1)$
\end{itemize}

\textbf{Total:}
\begin{itemize}
  \item Sumas de Cuadrados: $VT$ = $VE($t$)$ + $VE(B)$ + $VE($tB$)$ + $VNE$
  \item grados de Libertad: $N - 1$
\end{itemize}

De tal modo que la hip\'otesis nula de igualdad de los efectos del factor A se rechaza (a un nivel de confianza del 100(1 - $alfa$)\% si\\

$F_t$ $>$ $F_\ alfa,(K-1),kn(m-1)$\\

De tal modo que la hip\'otesis nula de igualdad de los efectos del factor B se rechaza (a un nivel de confianza del 100(1 - $alfa$)\% si\\

$F_B$ $>$ $F_\ alfa,(n-1),kn(m-1)$\\

De tal modo que la hip\'otesis nula de igualdad de los efectos del factor A y B se rechaza (a un nivel de confianza del 100(1 - $alfa$)\% si\\

$F_\ tB$ $>$ $F_\ alfa,(k-1)(n-1),kn(m-1)$\\

\section{EJEMPLO 1.}

Se llev\'o a cabo un estudio del efecto de la temperatura sobre el porcentaje de encogimiento de telas te\~nidas, con dos r\'eplicas para cada uno de cuatro tipos de tela en un dise\~no totalmente aleatorizado. Los datos son el porcentaje de encogimiento de dos r\'eplicas de tela secadas a cuatro temperaturas; los cuales se muestran a continuaci\'on. 

\begin{knitrout}
\definecolor{shadecolor}{rgb}{0.969, 0.969, 0.969}\color{fgcolor}\begin{kframe}
\begin{alltt}
\hlcom{# Definiendo el vector que contendr\textbackslash{}'a el factor A. El primer novillo es}
\hlcom{# asignado al bloque 1, el siguiente al 2, el tercero al 3 el cuarto al 4,}
\hlcom{# y se inicia el ciclo. }

\hlstd{FactorA} \hlkwb{<-} \hlkwd{gl}\hlstd{(}\hlkwc{n}\hlstd{=}\hlnum{4}\hlstd{,} \hlkwc{k}\hlstd{=}\hlnum{8}\hlstd{,} \hlkwc{length}\hlstd{=}\hlnum{32}\hlstd{);FactorA}
\end{alltt}
\begin{verbatim}
##  [1] 1 1 1 1 1 1 1 1 2 2 2 2 2 2 2 2 3 3 3 3 3 3 3 3 4 4 4 4 4 4 4 4
## Levels: 1 2 3 4
\end{verbatim}
\begin{alltt}
\hlcom{# k=8 especifica que se introducir\textbackslash{}'an las observaciones en orden de cada}
\hlcom{# factor. Para mayor comodidad se introducir\textbackslash{}'an los datos en orden de}
\hlcom{# celda. }

\hlcom{# Se crea el vector que contendr\textbackslash{}'a los tratamientos de los novillos}
\hlcom{# (raciones de alimento) los primeros cuatros se les asigna el tratamiento}
\hlcom{# 1, los siguientes cuatro el 2, y as\textbackslash{}'i sucesivamente.}

\hlstd{FactorB}\hlkwb{<-} \hlkwd{gl}\hlstd{(}\hlkwc{n}\hlstd{=}\hlnum{4}\hlstd{,} \hlkwc{k}\hlstd{=}\hlnum{2}\hlstd{,}\hlkwc{length}\hlstd{=}\hlnum{32}\hlstd{);FactorB}
\end{alltt}
\begin{verbatim}
##  [1] 1 1 2 2 3 3 4 4 1 1 2 2 3 3 4 4 1 1 2 2 3 3 4 4 1 1 2 2 3 3 4 4
## Levels: 1 2 3 4
\end{verbatim}
\begin{alltt}
\hlcom{# k=2 especifica que las primeras dos observaciones ser\textbackslash{}'an correspondien}
\hlcom{# tes al primer nivel, las dos siguientes al segundo, y as\textbackslash{}'is }
\hlcom{# sucesivamente. Cuando se finalice en el cuarto nivel se iniciar\textbackslash{}'a }
\hlcom{# nuevamente en el nivel 1. }

\hlcom{# Se digitan los pesos de los novillos }

\hlstd{Porcentaje} \hlkwb{<-} \hlkwd{c}\hlstd{(}\hlnum{1.8}\hlstd{,} \hlnum{2.1}\hlstd{,} \hlnum{2.0}\hlstd{,} \hlnum{2.1}\hlstd{,} \hlnum{4.6}\hlstd{,} \hlnum{5.0}\hlstd{,} \hlnum{7.5}\hlstd{,} \hlnum{7.9}\hlstd{,} \hlnum{2.2}\hlstd{,} \hlnum{2.4}\hlstd{,}
                \hlnum{4.2}\hlstd{,} \hlnum{4.0}\hlstd{,} \hlnum{5.4}\hlstd{,} \hlnum{5.6}\hlstd{,} \hlnum{9.8}\hlstd{,} \hlnum{9.2}\hlstd{,} \hlnum{2.8}\hlstd{,} \hlnum{3.2}\hlstd{,} \hlnum{4.4}\hlstd{,} \hlnum{4.8}\hlstd{,}
                \hlnum{8.7}\hlstd{,} \hlnum{8.4}\hlstd{,} \hlnum{13.2}\hlstd{,} \hlnum{13.0}\hlstd{,} \hlnum{3.2}\hlstd{,} \hlnum{3.6}\hlstd{,} \hlnum{3.3}\hlstd{,} \hlnum{3.5}\hlstd{,} \hlnum{5.7}\hlstd{,} \hlnum{5.8}\hlstd{,}
                \hlnum{10.9}\hlstd{,} \hlnum{11.1}\hlstd{)}
\hlstd{Porcentaje}
\end{alltt}
\begin{verbatim}
##  [1]  1.8  2.1  2.0  2.1  4.6  5.0  7.5  7.9  2.2  2.4  4.2  4.0  5.4  5.6
## [15]  9.8  9.2  2.8  3.2  4.4  4.8  8.7  8.4 13.2 13.0  3.2  3.6  3.3  3.5
## [29]  5.7  5.8 10.9 11.1
\end{verbatim}
\begin{alltt}
\hlcom{# Se registra en una hoja de datos los resultados del experimento }

\hlstd{datos3} \hlkwb{<-} \hlkwd{data.frame}\hlstd{(}\hlkwc{FactorA} \hlstd{= FactorA,} \hlkwc{FactorB} \hlstd{= FactorB,} \hlkwc{Porcentaje}\hlstd{=Porcentaje)}
\hlstd{datos3}
\end{alltt}
\begin{verbatim}
##    FactorA FactorB Porcentaje
## 1        1       1        1.8
## 2        1       1        2.1
## 3        1       2        2.0
## 4        1       2        2.1
## 5        1       3        4.6
## 6        1       3        5.0
## 7        1       4        7.5
## 8        1       4        7.9
## 9        2       1        2.2
## 10       2       1        2.4
## 11       2       2        4.2
## 12       2       2        4.0
## 13       2       3        5.4
## 14       2       3        5.6
## 15       2       4        9.8
## 16       2       4        9.2
## 17       3       1        2.8
## 18       3       1        3.2
## 19       3       2        4.4
## 20       3       2        4.8
## 21       3       3        8.7
## 22       3       3        8.4
## 23       3       4       13.2
## 24       3       4       13.0
## 25       4       1        3.2
## 26       4       1        3.6
## 27       4       2        3.3
## 28       4       2        3.5
## 29       4       3        5.7
## 30       4       3        5.8
## 31       4       4       10.9
## 32       4       4       11.1
\end{verbatim}
\end{kframe}
\end{knitrout}

Utilizando un nivel de significancia del 5\%, contraste el siguiente conjunto de hip\'otesis:
\begin{itemize}
  \item Las hip\'otesis a contrastar para el factor A son:\\ 
  $H_o$: $t_1$ = $t_2$ = $t_3$ = $t_4$\\
  $H_1$: $t_i$ distinto 0; para al menos un $i$
  \item Las hip\'otesis a contrastar para el factor B son:\\
  $H_o$: $B_1$ = $B_2$ = $B_3$ = $B_4$\\
  $H_1$: $B_j$ distinto 0; para al menos un $j$
  \item Las hip\'otesis a contrastar para la interacci\'on de los factores A y B son:\\
  $H_o$: $($tB$)_ \ij$ = 0; para todo i,j\\
  $H_1$: $($tB$)_ \ij$ disntinto 0 para al menos una combinaci\'on ij
  \item Ejecutar el script "anova3.R"

\begin{knitrout}
\definecolor{shadecolor}{rgb}{0.969, 0.969, 0.969}\color{fgcolor}\begin{kframe}
\begin{alltt}
\hlcom{# Definiendo el vector que contendr\textbackslash{}'a el factor A. El primer novillo es}
\hlcom{# asignado al bloque 1, el siguiente al 2, el tercero al 3 el cuarto al 4,}
\hlcom{# y se inicia el ciclo. }

\hlstd{FactorA} \hlkwb{<-} \hlkwd{gl}\hlstd{(}\hlkwc{n}\hlstd{=}\hlnum{4}\hlstd{,} \hlkwc{k}\hlstd{=}\hlnum{8}\hlstd{,} \hlkwc{length}\hlstd{=}\hlnum{32}\hlstd{);FactorA}
\end{alltt}
\begin{verbatim}
##  [1] 1 1 1 1 1 1 1 1 2 2 2 2 2 2 2 2 3 3 3 3 3 3 3 3 4 4 4 4 4 4 4 4
## Levels: 1 2 3 4
\end{verbatim}
\begin{alltt}
\hlcom{# k=8 especifica que se introducir\textbackslash{}'an las observaciones en orden de cada}
\hlcom{# factor. Para mayor comodidad se introducir\textbackslash{}'an los datos en orden de}
\hlcom{# celda. }

\hlcom{# Se crea el vector que contendr\textbackslash{}'a los tratamientos de los novillos}
\hlcom{# (raciones de alimento) los primeros cuatros se les asigna el tratamiento}
\hlcom{# 1, los siguientes cuatro el 2, y as\textbackslash{}'i sucesivamente.}

\hlstd{FactorB}\hlkwb{<-} \hlkwd{gl}\hlstd{(}\hlkwc{n}\hlstd{=}\hlnum{4}\hlstd{,} \hlkwc{k}\hlstd{=}\hlnum{2}\hlstd{,}\hlkwc{length}\hlstd{=}\hlnum{32}\hlstd{);FactorB}
\end{alltt}
\begin{verbatim}
##  [1] 1 1 2 2 3 3 4 4 1 1 2 2 3 3 4 4 1 1 2 2 3 3 4 4 1 1 2 2 3 3 4 4
## Levels: 1 2 3 4
\end{verbatim}
\begin{alltt}
\hlcom{# k=2 especifica que las primeras dos observaciones ser\textbackslash{}'an correspondien}
\hlcom{# tes al primer nivel, las dos siguientes al segundo, y as\textbackslash{}'is }
\hlcom{# sucesivamente. Cuando se finalice en el cuarto nivel se iniciar\textbackslash{}'a }
\hlcom{# nuevamente en el nivel 1. }

\hlcom{# Se digitan los pesos de los novillos }

\hlstd{Porcentaje} \hlkwb{<-} \hlkwd{c}\hlstd{(}\hlnum{1.8}\hlstd{,} \hlnum{2.1}\hlstd{,} \hlnum{2.0}\hlstd{,} \hlnum{2.1}\hlstd{,} \hlnum{4.6}\hlstd{,} \hlnum{5.0}\hlstd{,} \hlnum{7.5}\hlstd{,} \hlnum{7.9}\hlstd{,} \hlnum{2.2}\hlstd{,} \hlnum{2.4}\hlstd{,}
                \hlnum{4.2}\hlstd{,} \hlnum{4.0}\hlstd{,} \hlnum{5.4}\hlstd{,} \hlnum{5.6}\hlstd{,} \hlnum{9.8}\hlstd{,} \hlnum{9.2}\hlstd{,} \hlnum{2.8}\hlstd{,} \hlnum{3.2}\hlstd{,} \hlnum{4.4}\hlstd{,} \hlnum{4.8}\hlstd{,}
                \hlnum{8.7}\hlstd{,} \hlnum{8.4}\hlstd{,} \hlnum{13.2}\hlstd{,} \hlnum{13.0}\hlstd{,} \hlnum{3.2}\hlstd{,} \hlnum{3.6}\hlstd{,} \hlnum{3.3}\hlstd{,} \hlnum{3.5}\hlstd{,} \hlnum{5.7}\hlstd{,} \hlnum{5.8}\hlstd{,}
                \hlnum{10.9}\hlstd{,} \hlnum{11.1}\hlstd{)}
\hlstd{Porcentaje}
\end{alltt}
\begin{verbatim}
##  [1]  1.8  2.1  2.0  2.1  4.6  5.0  7.5  7.9  2.2  2.4  4.2  4.0  5.4  5.6
## [15]  9.8  9.2  2.8  3.2  4.4  4.8  8.7  8.4 13.2 13.0  3.2  3.6  3.3  3.5
## [29]  5.7  5.8 10.9 11.1
\end{verbatim}
\begin{alltt}
\hlcom{# Se registra en una hoja de datos los resultados del experimento }

\hlstd{datos3} \hlkwb{<-} \hlkwd{data.frame}\hlstd{(}\hlkwc{FactorA} \hlstd{= FactorA,} \hlkwc{FactorB} \hlstd{= FactorB,} \hlkwc{Porcentaje}\hlstd{=Porcentaje)}
\hlstd{datos3}
\end{alltt}
\begin{verbatim}
##    FactorA FactorB Porcentaje
## 1        1       1        1.8
## 2        1       1        2.1
## 3        1       2        2.0
## 4        1       2        2.1
## 5        1       3        4.6
## 6        1       3        5.0
## 7        1       4        7.5
## 8        1       4        7.9
## 9        2       1        2.2
## 10       2       1        2.4
## 11       2       2        4.2
## 12       2       2        4.0
## 13       2       3        5.4
## 14       2       3        5.6
## 15       2       4        9.8
## 16       2       4        9.2
## 17       3       1        2.8
## 18       3       1        3.2
## 19       3       2        4.4
## 20       3       2        4.8
## 21       3       3        8.7
## 22       3       3        8.4
## 23       3       4       13.2
## 24       3       4       13.0
## 25       4       1        3.2
## 26       4       1        3.6
## 27       4       2        3.3
## 28       4       2        3.5
## 29       4       3        5.7
## 30       4       3        5.8
## 31       4       4       10.9
## 32       4       4       11.1
\end{verbatim}
\begin{alltt}
\hlcom{# Se aplica el análisis de varianza }

\hlstd{mod3} \hlkwb{<-} \hlkwd{aov}\hlstd{(Porcentaje} \hlopt{~} \hlstd{FactorA} \hlopt{*} \hlstd{FactorB,} \hlkwc{data} \hlstd{= datos3)}

\hlcom{# Observe con el signo * se indican c\textbackslash{}'omo se descompone la varianza; es}
\hlcom{# decir, ser\textbackslash{}'a el efecto del Factor A m\textbackslash{}'as el efecto del Factor B m\textbackslash{}'as}
\hlcom{# el efecto de la interacci\textbackslash{}'on de los factores A y B (* indica que tome}
\hlcom{# en cuenta los efectos principales m\textbackslash{}'as efecto de la interacci\textbackslash{}'on).}

\hlcom{# Se muestra la tabla ANOVA del experimento }

\hlkwd{summary}\hlstd{(mod3)}
\end{alltt}
\begin{verbatim}
##                 Df Sum Sq Mean Sq F value   Pr(>F)    
## FactorA          3  41.88   13.96  279.18 5.05e-14 ***
## FactorB          3 283.94   94.65 1892.91  < 2e-16 ***
## FactorA:FactorB  9  15.86    1.76   35.24 7.09e-09 ***
## Residuals       16   0.80    0.05                     
## ---
## Signif. codes:  0 '***' 0.001 '**' 0.01 '*' 0.05 '.' 0.1 ' ' 1
\end{verbatim}
\begin{alltt}
\hlcom{# Note que seg\textbackslash{}'un los resultados, rechazamos cada una de las hip\textbackslash{}'otesis. }
\end{alltt}
\end{kframe}
\end{knitrout}



\end{itemize}
\end{document}
