\documentclass[12pt,letterpaper]{article}\usepackage[]{graphicx}\usepackage[]{color}
%% maxwidth is the original width if it is less than linewidth
%% otherwise use linewidth (to make sure the graphics do not exceed the margin)
\makeatletter
\def\maxwidth{ %
  \ifdim\Gin@nat@width>\linewidth
    \linewidth
  \else
    \Gin@nat@width
  \fi
}
\makeatother

\definecolor{fgcolor}{rgb}{0.345, 0.345, 0.345}
\newcommand{\hlnum}[1]{\textcolor[rgb]{0.686,0.059,0.569}{#1}}%
\newcommand{\hlstr}[1]{\textcolor[rgb]{0.192,0.494,0.8}{#1}}%
\newcommand{\hlcom}[1]{\textcolor[rgb]{0.678,0.584,0.686}{\textit{#1}}}%
\newcommand{\hlopt}[1]{\textcolor[rgb]{0,0,0}{#1}}%
\newcommand{\hlstd}[1]{\textcolor[rgb]{0.345,0.345,0.345}{#1}}%
\newcommand{\hlkwa}[1]{\textcolor[rgb]{0.161,0.373,0.58}{\textbf{#1}}}%
\newcommand{\hlkwb}[1]{\textcolor[rgb]{0.69,0.353,0.396}{#1}}%
\newcommand{\hlkwc}[1]{\textcolor[rgb]{0.333,0.667,0.333}{#1}}%
\newcommand{\hlkwd}[1]{\textcolor[rgb]{0.737,0.353,0.396}{\textbf{#1}}}%

\usepackage{framed}
\makeatletter
\newenvironment{kframe}{%
 \def\at@end@of@kframe{}%
 \ifinner\ifhmode%
  \def\at@end@of@kframe{\end{minipage}}%
  \begin{minipage}{\columnwidth}%
 \fi\fi%
 \def\FrameCommand##1{\hskip\@totalleftmargin \hskip-\fboxsep
 \colorbox{shadecolor}{##1}\hskip-\fboxsep
     % There is no \\@totalrightmargin, so:
     \hskip-\linewidth \hskip-\@totalleftmargin \hskip\columnwidth}%
 \MakeFramed {\advance\hsize-\width
   \@totalleftmargin\z@ \linewidth\hsize
   \@setminipage}}%
 {\par\unskip\endMakeFramed%
 \at@end@of@kframe}
\makeatother

\definecolor{shadecolor}{rgb}{.97, .97, .97}
\definecolor{messagecolor}{rgb}{0, 0, 0}
\definecolor{warningcolor}{rgb}{1, 0, 1}
\definecolor{errorcolor}{rgb}{1, 0, 0}
\newenvironment{knitrout}{}{} % an empty environment to be redefined in TeX

\usepackage{alltt}
 \usepackage[left=2cm,right=2cm,top=2cm,bottom=2cm]{geometry}
\usepackage[ansinew]{inputenc}
\usepackage[spanish]{babel}
\usepackage{amsmath}
\usepackage{amsfonts}
\usepackage{amssymb}
\usepackage{dsfont}
\usepackage{multicol} 
\usepackage{subfigure}
\usepackage{graphicx}
\usepackage{float} 
\usepackage{verbatim} 
\usepackage[left=2cm,right=2cm,top=2cm,bottom=2cm]{geometry}
\usepackage{fancyhdr}
\pagestyle{fancy} 
\fancyhead[LO]{\leftmark}
\usepackage{caption}
\newtheorem{definicion}{Definci\'on}
\IfFileExists{upquote.sty}{\usepackage{upquote}}{}
\begin{document}

\begin{titlepage}
\setlength{\unitlength}{1 cm} %Especificar unidad de trabajo


\begin{center}
\textbf{{\large UNIVERSIDAD DE EL SALVADOR}\\
{\large FACULTAD MULTIDISCIPLINARIA DE OCCIDENTE}\\
{\large DEPARTAMENTO DE MATEM\'ATICA}}\\[0.50 cm]

\begin{picture}(18,4)
 \put(7,0){\includegraphics[width=4cm]{minerva.jpg}}
\end{picture}
\\[0.25 cm]

\textbf{{\large Licenciatura en Estad\'istica}\\[1.25cm]
{\large Control Estadistico del Paquete R }\\[2 cm]
%\setlength{\unitlength}{1 cm}
{\large  \textbf{''UNIDAD CUATRO"}}\\[3 cm]
{\large Alumna:}\\
{\large Erika Beatr'iz Guill\'en Pineda}\\[2cm]
{\large Fecha de elaboraci\'on}\\
Santa Ana - \today }
\end{center}
\end{titlepage}

\newtheorem{teorema}{Teorema}
\newtheorem{prop}{Proposici\'on}[section]


\lhead{PR\'ACTICA 18}
\lfoot{LICENCIATURA EN ESTAD\'ISTICA}
\cfoot{UESOCC}
\rfoot{\thepage}
%\pagestyle{fancy} 

\setcounter{page}{1}
\newpage


\section{INTERVALOS DE CONFIANZA PARA UNA MEDIAPOBLACIONAL.}

\textbf{PRIMER CASO: VARIANZA CONOCIDA}


\begin{knitrout}
\definecolor{shadecolor}{rgb}{0.969, 0.969, 0.969}\color{fgcolor}\begin{kframe}
\begin{alltt}
\hlcom{# Define una variable n para poder cambiar facilmente el tama\textbackslash{}~no de la muestra}

\hlstd{n} \hlkwb{=} \hlnum{10}

\hlcom{# Define una variable para alfa }

\hlstd{alfa} \hlkwb{=} \hlnum{0.95}

\hlcom{# Genera dos conjuntos de n muestras aleatorias }
\hlcom{# Ambos siguen una distribuci\textbackslash{}'on normal de parametros }

\hlstd{mu} \hlkwb{=} \hlnum{174.7}
\hlstd{sigma} \hlkwb{=} \hlnum{1.5}
\hlstd{x} \hlkwb{=} \hlkwd{rnorm}\hlstd{(n,} \hlnum{174.7}\hlstd{,} \hlnum{1.5}\hlstd{)}
\hlstd{x2} \hlkwb{=} \hlkwd{rnorm}\hlstd{(n,} \hlnum{174.7}\hlstd{,} \hlnum{1.5}\hlstd{)}

\hlstd{sigma} \hlkwb{=} \hlnum{1.5}
\hlstd{media} \hlkwb{=} \hlnum{174.7}
\hlstd{margenError} \hlkwb{=} \hlkwd{qnorm}\hlstd{(alfa}\hlopt{/}\hlnum{2}\hlstd{,} \hlkwc{mean}\hlstd{=}\hlnum{0}\hlstd{,} \hlkwc{sd}\hlstd{=}\hlnum{1}\hlstd{,} \hlkwc{lower.tail} \hlstd{=} \hlnum{FALSE}\hlstd{)}\hlopt{*}\hlstd{sigma}\hlopt{/}\hlkwd{sqrt}\hlstd{(n)}
\hlstd{extrIzq} \hlkwb{=} \hlstd{media} \hlopt{-} \hlstd{margenError}
\hlstd{extrDer} \hlkwb{=} \hlstd{media} \hlopt{+} \hlstd{margenError}
\hlkwd{print}\hlstd{(}\hlkwd{c}\hlstd{(extrIzq, extrDer))}
\end{alltt}
\begin{verbatim}
## [1] 174.6703 174.7297
\end{verbatim}
\end{kframe}
\end{knitrout}

\begin{center}
\textbf{CASO DOS: VARIANZA DESCONOCIDA}
\end{center}

\begin{knitrout}
\definecolor{shadecolor}{rgb}{0.969, 0.969, 0.969}\color{fgcolor}\begin{kframe}
\begin{alltt}
\hlstd{media} \hlkwb{<-} \hlkwa{function}\hlstd{(}\hlkwc{x}\hlstd{)}
\hlstd{\{}

  \hlstd{n} \hlkwb{=} \hlkwd{length}\hlstd{(x)}
\hlstd{suma} \hlkwb{<-} \hlnum{0.0}
\hlkwa{for}\hlstd{(i} \hlkwa{in} \hlnum{1}\hlopt{:}\hlstd{n) suma} \hlkwb{=} \hlstd{suma} \hlopt{+} \hlstd{x[i]}
\hlstd{media} \hlkwb{=} \hlstd{suma}\hlopt{/}\hlstd{n}

\hlstd{\}}
\hlkwd{save}\hlstd{(media,} \hlkwc{file}\hlstd{=} \hlstr{"media.RData"}\hlstd{)}
\hlkwd{rm}\hlstd{(}\hlkwc{list}\hlstd{=}\hlkwd{ls}\hlstd{(}\hlkwc{all}\hlstd{=}\hlnum{TRUE}\hlstd{))}
\hlkwd{load}\hlstd{(}\hlstr{"media.RData"}\hlstd{)}

\hlstd{z} \hlkwb{<-}\hlkwd{c}\hlstd{(}\hlnum{3.4}\hlstd{,} \hlnum{2.8}\hlstd{,} \hlnum{4.4}\hlstd{,} \hlnum{2.5}\hlstd{,} \hlnum{3.3}\hlstd{,}\hlnum{4.0}\hlstd{,} \hlnum{4.8}\hlstd{,} \hlnum{2.9}\hlstd{,} \hlnum{5.6}\hlstd{,} \hlnum{5.2}\hlstd{,} \hlnum{3.7}\hlstd{,} \hlnum{3.0}\hlstd{,} \hlnum{3.6}\hlstd{,} \hlnum{2.8}\hlstd{,} \hlnum{4.8}\hlstd{);}
\hlstd{(}\hlkwd{media}\hlstd{(z))}
\end{alltt}
\begin{verbatim}
## [1] 3.786667
\end{verbatim}
\begin{alltt}
\hlstd{(}\hlkwd{sd}\hlstd{(z))}
\end{alltt}
\begin{verbatim}
## [1] 0.9709102
\end{verbatim}
\begin{alltt}
\hlkwd{t.test}\hlstd{(z,} \hlkwc{conf.level} \hlstd{=} \hlnum{0.95}\hlstd{)}
\end{alltt}
\begin{verbatim}
## 
## 	One Sample t-test
## 
## data:  z
## t = 15.105, df = 14, p-value = 4.64e-10
## alternative hypothesis: true mean is not equal to 0
## 95 percent confidence interval:
##  3.248995 4.324339
## sample estimates:
## mean of x 
##  3.786667
\end{verbatim}
\end{kframe}
\end{knitrout}

\begin{center}
\textbf{INTERVALOS DE CONFIANZA PARA UNA PROPORCI\'ON.}
\end{center}

\begin{knitrout}
\definecolor{shadecolor}{rgb}{0.969, 0.969, 0.969}\color{fgcolor}\begin{kframe}
\begin{alltt}
\hlcom{# Utilizando la funci\textbackslash{}'on proporcionada por el R.}

\hlkwd{prop.test}\hlstd{(}\hlnum{360}\hlstd{,} \hlnum{1200}\hlstd{,} \hlkwc{alternative} \hlstd{=} \hlstr{"two.sided"}\hlstd{,} \hlkwc{conf.level}\hlstd{=}\hlnum{0.95}\hlstd{)}
\end{alltt}
\begin{verbatim}
## 
## 	1-sample proportions test with continuity correction
## 
## data:  360 out of 1200, null probability 0.5
## X-squared = 191.2, df = 1, p-value < 2.2e-16
## alternative hypothesis: true p is not equal to 0.5
## 95 percent confidence interval:
##  0.2743388 0.3269581
## sample estimates:
##   p 
## 0.3
\end{verbatim}
\end{kframe}
\end{knitrout}

\begin{knitrout}
\definecolor{shadecolor}{rgb}{0.969, 0.969, 0.969}\color{fgcolor}\begin{kframe}
\begin{alltt}
\hlcom{# Creando nuestra propia funci\textbackslash{}'on}

\hlstd{intervaloProp} \hlkwb{<-} \hlkwa{function}\hlstd{(}\hlkwc{x}\hlstd{,} \hlkwc{n}\hlstd{,} \hlkwc{nivel.conf} \hlstd{=} \hlnum{0.95}\hlstd{)}

\hlstd{\{}
\hlstd{pe} \hlkwb{<-} \hlstd{x}\hlopt{/}\hlstd{n}
\hlstd{alfa} \hlkwb{<-} \hlstd{(}\hlnum{1} \hlopt{-} \hlstd{nivel.conf)}
\hlstd{z} \hlkwb{<-} \hlkwd{qnorm}\hlstd{(}\hlnum{1}\hlopt{-}\hlstd{alfa}\hlopt{/}\hlnum{2}\hlstd{)}
\hlstd{SE} \hlkwb{<-} \hlkwd{sqrt}\hlstd{(pe}\hlopt{*}\hlstd{(}\hlnum{1}\hlopt{-}\hlstd{pe)}\hlopt{/}\hlstd{n)}
  \hlkwd{print}\hlstd{(}\hlkwd{rbind}\hlstd{(pe, alfa, z, SE))}
\hlstd{LInf} \hlkwb{<-} \hlstd{pe}\hlopt{-}\hlstd{z}\hlopt{*}\hlstd{SE}
\hlstd{LSup} \hlkwb{<-} \hlstd{pe}\hlopt{+}\hlstd{z}\hlopt{*}\hlstd{SE}
\hlkwd{print}\hlstd{(}\hlstr{" "}\hlstd{)}
  \hlkwd{print}\hlstd{(}\hlkwd{paste}\hlstd{(}\hlstr{"Intervalo para p es: ["}\hlstd{,} \hlkwd{round}\hlstd{(LInf,} \hlnum{2}\hlstd{),}
              \hlstr{","}\hlstd{,} \hlkwd{round}\hlstd{(LSup,} \hlnum{2}\hlstd{),} \hlstr{"]"}\hlstd{))}
\hlstd{\}}
\hlstd{x}\hlkwb{=}\hlnum{360}\hlstd{; n}\hlkwb{=}\hlnum{1200}\hlstd{; nivel.conf}\hlkwb{=}\hlnum{0.95}

\hlkwd{intervaloProp}\hlstd{(x, n, nivel.conf)}
\end{alltt}
\begin{verbatim}
##            [,1]
## pe   0.30000000
## alfa 0.05000000
## z    1.95996398
## SE   0.01322876
## [1] " "
## [1] "Intervalo para p es: [ 0.27 , 0.33 ]"
\end{verbatim}
\end{kframe}
\end{knitrout}

\begin{center}
\textbf{INTERVALOS DE CONFIANZA PARA LA VARIANZA POBLACIONAL}
\end{center}

\begin{knitrout}
\definecolor{shadecolor}{rgb}{0.969, 0.969, 0.969}\color{fgcolor}\begin{kframe}
\begin{alltt}
\hlstd{h} \hlkwb{<-} \hlkwd{c} \hlstd{(}\hlnum{46.4}\hlstd{,} \hlnum{46.1}\hlstd{,} \hlnum{45.8}\hlstd{,} \hlnum{47.0}\hlstd{,} \hlnum{46.1}\hlstd{,} \hlnum{45.9}\hlstd{,} \hlnum{45.8}\hlstd{,} \hlnum{46.9}\hlstd{,} \hlnum{45.2}\hlstd{,} \hlnum{46.0}\hlstd{);}
\hlstd{h}
\end{alltt}
\begin{verbatim}
##  [1] 46.4 46.1 45.8 47.0 46.1 45.9 45.8 46.9 45.2 46.0
\end{verbatim}
\begin{alltt}
\hlcom{# Creando nuestra propia funci\textbackslash{}'on}

\hlstd{intervalovarpobla} \hlkwb{<-} \hlkwa{function}\hlstd{(}\hlkwc{n}\hlstd{,} \hlkwc{nivel.conf}\hlstd{=}\hlnum{0.95}\hlstd{)}
\hlstd{\{}
  \hlstd{varianza} \hlkwb{=} \hlstd{(}\hlkwd{var}\hlstd{(h))} \hlcom{# varianza}
  \hlstd{gl} \hlkwb{=} \hlstd{n} \hlopt{-} \hlnum{1} \hlcom{# grados de libertad}
  \hlstd{alfa} \hlkwb{=} \hlnum{1} \hlopt{-} \hlstd{nivel.conf} \hlcom{# valor de alfa}
  \hlstd{X_1} \hlkwb{<-} \hlkwd{qnorm}\hlstd{(}\hlnum{1}\hlopt{-}\hlstd{(alfa}\hlopt{/}\hlnum{2}\hlstd{), gl)}
  \hlstd{X_2} \hlkwb{<-} \hlkwd{qnorm}\hlstd{((alfa}\hlopt{/}\hlnum{2}\hlstd{), gl)}
  \hlstd{u} \hlkwb{<-} \hlstd{((n} \hlopt{-} \hlnum{1}\hlstd{)}\hlopt{*}\hlstd{varianza)}
  \hlkwd{print}\hlstd{(}\hlkwd{rbind}\hlstd{(varianza, gl, alfa, X_1, X_2, u))}
  \hlstd{LInf} \hlkwb{<-} \hlstd{u}\hlopt{/}\hlstd{X_1}
  \hlstd{LSup} \hlkwb{<-} \hlstd{u}\hlopt{/}\hlstd{X_2}
\hlkwd{print}\hlstd{(}\hlstr{" "}\hlstd{)}
  \hlkwd{print}\hlstd{(}\hlkwd{paste}\hlstd{(}\hlstr{"Intervalo para D es: ["}\hlstd{,} \hlkwd{round}\hlstd{(LInf,} \hlnum{2}\hlstd{),} \hlstr{","}\hlstd{,} \hlkwd{round}\hlstd{(LSup,} \hlnum{2}\hlstd{),} \hlstr{"]"}\hlstd{))}

\hlstd{\}}
 \hlstd{n}\hlkwb{=}\hlnum{10}\hlstd{; nivel.conf}\hlkwb{=}\hlnum{0.95}
\hlkwd{intervalovarpobla} \hlstd{( n, nivel.conf)}
\end{alltt}
\begin{verbatim}
##                [,1]
## varianza  0.2862222
## gl        9.0000000
## alfa      0.0500000
## X_1      10.9599640
## X_2       7.0400360
## u         2.5760000
## [1] " "
## [1] "Intervalo para D es: [ 0.24 , 0.37 ]"
\end{verbatim}
\end{kframe}
\end{knitrout}


\end{document}
